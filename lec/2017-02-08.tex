\documentclass[12pt, leqno]{article}
\input{common}

\begin{document}
\hdr{2017-02-06}

\section{Introduction}

For the next few lectures, we will build tools to solve linear
systems.  Our main tool will be the factorization $PA = LU$, where $P$
is a permutation, $L$ is a unit lower triangular matrix, and $U$ is an
upper triangular matrix.  As we will see, the Gaussian elimination
algorithm learned in a first linear algebra class implicitly computes
this decomposition; but by thinking about the decomposition
explicitly, we find other ways to organize the computation.

\section{Triangular solves}

Suppose that we have computed a factorization $PA = LU$.  How can we
use this to solve a linear system of the form $Ax = b$?  Permuting the
rows of $A$ and $b$, we have
\[
  PAx = LUx = Pb,
\]
and therefore
\[
  x = U^{-1} L^{-1} Pb.
\]
So we can reduce the problem of finding $x$ to two simpler problems:
\begin{enumerate}
\item
  Solve $Ly = Pb$
\item
  Solve $Ux = y$
\end{enumerate}
We assume the matrix $L$ is unit lower triangular (diagonal of all
ones + lower triangular), and $U$ is upper triangular, so we can solve
linear systems with $L$ and $U$ involving forward and backward
substitution.

As a concrete example, suppose
\[
  L = \begin{bmatrix}
        1 & 0 & 0 \\
        2 & 1 & 0 \\
        3 & 2 & 1
      \end{bmatrix}, \quad
  d = \begin{bmatrix} 1 \\ 1 \\ 3 \end{bmatrix}
\]
To solve a linear system of the form $Ly = d$, we process each row in
turn to find the value of the corresponding entry of $y$:
\begin{enumerate}
\item Row 1:
  $y_1 = d_1$
\item Row 2: $2y_1 + y_2 = d_2$, or $y_2 = d_2 - 2y_1$
\item Row 3: $3y_1 + 2 y_2 + y_3 = d_3$, or $y_3 = d_3 - 3y_1 - 2y_2$
\end{enumerate}
More generally, the {\em forward substitution} algorithm for solving
unit lower triangular linear systems $Ly = d$ looks like
\begin{lstlisting}
  y = d;
  for i=2:n
    y(i) = d(i)-L(1:i-1)*y(1:i-1)
  end
\end{lstlisting}
Similarly, there is a {\em backward substitution} algorithm for
solving upper triangular linear systems $Ux = d$
\begin{lstlisting}
  x(n) = d(n)/U(n,n);
  for i=n-1:-1:1
    x(i) = ( d(i)-U(i+1:n)*x(i+1:n) )/U(i,i)
  end
\end{lstlisting}
Each of these algorithms takes $O(n^2)$ time.

\section{Gaussian elimination by example}

Let's start our discussion of $LU$ factorization by working through
these ideas with a concrete example:
\[
  A =
  \begin{bmatrix}
    1 & 4 & 7 \\
    2 & 5 & 8 \\
    3 & 6 & 10
  \end{bmatrix}.
\]
To eliminate the subdiagonal entries $a_{21}$ and $a_{31}$, we
subtract twice the first row from the second row, and thrice the
first row from the third row:
\[
  A^{(1)} =
  \begin{bmatrix}
    1 & 4 & 7 \\
    2 & 5 & 8 \\
    3 & 6 & 10
  \end{bmatrix} -
  \begin{bmatrix}
    0 \cdot 1 & 0 \cdot 4 & 0 \cdot 7 \\
    2 \cdot 1 & 2 \cdot 4 & 2 \cdot 7 \\
    3 \cdot 1 & 3 \cdot 4 & 3 \cdot 7
  \end{bmatrix}
  =
  \begin{bmatrix}
    1 &  4 &  7 \\
    0 & -3 & -6 \\
    0 & -6 & -11
  \end{bmatrix}.
\]
That is, the step comes from a rank-1 update to the matrix:
\[
  A^{(1)} =
  A -
  \begin{bmatrix} 0 \\ 2 \\ 3 \end{bmatrix}
  \begin{bmatrix} 1 & 4 & 7 \end{bmatrix}.
\]
Another way to think of this step is as a linear transformation
$A^{(1)} = M_1 A$, where the rows of $M_1$ describe the multiples
of rows of the original matrix that go into rows of the updated matrix:
\[
  M_1 = \begin{bmatrix} 1 & 0 & 0 \\ -2 & 1 & 0 \\ -3 & 0 & 1 \end{bmatrix}
      = I - \begin{bmatrix} 0 \\ 2 \\ 3 \end{bmatrix}
            \begin{bmatrix} 1 & 0 & 0 \end{bmatrix}
      = I - \tau_1 e_1^T.
\]
Similarly, in the second step of the algorithm, we subtract twice the second
row from the third row:
\[
  \begin{bmatrix}
    1 &  4 &  7 \\
    0 & -3 & -6 \\
    0 &  0 &  1
  \end{bmatrix} =
  \begin{bmatrix}
    1 & 0 & 0 \\
    0 & 1 & 0 \\
    0 & -2 & 1
  \end{bmatrix}
  \begin{bmatrix}
    1 &  4 &  7 \\
    0 & -3 & -6 \\
    0 & -6 & -11
  \end{bmatrix} =
  \left( I - \begin{bmatrix} 0 \\ 0 \\ 2 \end{bmatrix}
             \begin{bmatrix} 0 & 1 & 0 \end{bmatrix} \right) A^{(1)}.
\]
More compactly: $U = (I-\tau_2 e_2^T) A^{(1)}$.

Putting everything together, we have computed
\[
  U = (I-\tau_2 e_2^T) (I-\tau_1 e_1^T) A.
\]
Therefore,
\[
  A = (I-\tau_1 e_1^T)^{-1} (I-\tau_2 e_2^T)^{-1} U = LU.
\]
Now, note that
\[
  (I-\tau_1 e_1^T) (I + \tau_1 e_1^T) =
  I - \tau_1 e_1^T + \tau_1 e_1^T - \tau_1 e_1^T \tau_1 e_1^T = I,
\]
since $e_1^T \tau_1$ (the first entry of $\tau_1$) is zero.  Therefore,
\[
  (I-\tau_1 e_1^T)^{-1} = (I+\tau_1 e_1^T)
\]
Similarly,
\[
  (I-\tau_2 e_2^T)^{-1} = (I+\tau_2 e_2^T)
\]
Thus,
\[
  L = (I+\tau_1 e_1^T)(I + \tau_2 e_2^T).
\]
Now, note that because $\tau_2$ is only nonzero in the third element,
$e_1^T \tau_2 = 0$; thus,
\begin{align*}
  L &= (I+\tau_1 e_1^T)(I + \tau_2 e_2^T) \\
    &= (I + \tau_1 e_1^T + \tau_2 e_2^T + \tau_1 (e_1^T \tau_2) e_2^T \\
    &= I + \tau_1 e_1^T + \tau_2 e_2^T \\
    &= \begin{bmatrix} 1 & 0 & 0 \\ 0 & 1 & 0 \\ 0 & 0 & 1 \end{bmatrix} +
       \begin{bmatrix} 0 & 0 & 0 \\ 2 & 0 & 0 \\ 3 & 0 & 0 \end{bmatrix} +
       \begin{bmatrix} 0 & 0 & 0 \\ 0 & 0 & 0 \\ 0 & 2 & 0 \end{bmatrix}
     = \begin{bmatrix} 1 & 0 & 0 \\ 2 & 1 & 0 \\ 3 & 2 & 1 \end{bmatrix}.
\end{align*}

The final factorization is
\[
A =
\begin{bmatrix} 1 & 4 & 7 \\ 2 & 5 & 8 \\ 3 & 6 & 10 \end{bmatrix} =
\begin{bmatrix} 1 & 0 & 0 \\ 2 & 1 & 0 \\ 3 & 2 & 1 \end{bmatrix}
\begin{bmatrix} 1 & 4 & 7 \\ 0 & -3 & -6 \\ 0 & 0 & 1 \end{bmatrix} = LU.
\]
The subdiagonal elements of $L$ are easy to read off: for $j
> i$, $l_{ij}$ is the multiple of row $j$ that we subtract from row
$i$ during elimination.  This means that it is easy to read off the
subdiagonal entries of $L$ during the elimination process.

\section{Basic LU factorization}

Let's generalize our previous algorithm and write a simple code for
$LU$ factorization.  We will leave the issue of pivoting to a later
discussion.  We'll start with a purely loop-based implementation:
\begin{lstlisting}
%
% Overwrites A with an upper triangular factor U, keeping track of
% multipliers in the matrix L.
%
function [L,A] = mylu(A)

  n = length(A);
  L = eye(n);
  for j=1:n-1
    for i=j+1:n

      % Figure out multiple of row j to subtract from row i
      L(i,j) = A(i,j)/A(j,j);

      % Subtract off the appropriate multiple
      A(i,j) = 0
      for k=j+1:n
        A(i,k) = A(i,k) - L(i,j)*A(j,k);
      end
    end
  end
\end{lstlisting}
We can write the two innermost loops more concisely in
terms of a {\em Gauss transformation} $M_j = I - \tau_j e_j^T$,
where $\tau_j$ is the vector of multipliers that appear when
eliminating in column $j$:
\begin{lstlisting}
%
% Overwrites A with an upper triangular factor U, keeping track of
% multipliers in the matrix L.
%
function [L,A] = mylu(A)

  n = length(A);
  L = eye(n);
  for j=1:n-1

    % Form vector of multipliers
    L(j+1:n,j) = A(j+1:n,j)/A(j,j);

    % Apply Gauss transformation
    A(j+1:n,j) = 0;
    A(j+1:n,j+1:n) = A(j+1:n,j+1:n)-L(j+1:n,j)*A(j,j+1:n);

  end
\end{lstlisting}

\newpage

\section{Problems to ponder}

\begin{enumerate}
\item
  What is the complexity of the Gaussian elimination algorithm?
\item
  Describe how to find $A^{-1}$ using Gaussian elimination.  Compare
  the cost of solving a linear system by computing and multiplying by
  $A^{-1}$ to the cost of doing Gaussian elimination and two
  triangular solves.
\item
  Consider a parallelipiped in $\bbR^{3}$ whose sides are given by
  the columns of a 3-by-3 matrix $A$.  Interpret $LU$ factorization
  geometrically, thinking of Gauss transformations as shearing
  operations.  Using the fact that shear transformations preserve
  volume, give a simple expression for tne volume of the parallelipiped.
\end{enumerate}

\end{document}
