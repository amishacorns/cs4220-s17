\documentclass[12pt, leqno]{article}
\input{common}

\begin{document}
\hdr{2017-02-10}

\section{Basic LU factorization}

In the last lecture, we wrote Gaussian elimination as a sequence
{\em Gauss transformations} $M_j = I - \tau_j e_j^T$, where $\tau_j$ is
the vector of multipliers that appear when eliminating in column $j$:
\begin{lstlisting}
%
% Overwrites A with an upper triangular factor U, keeping track of
% multipliers in the matrix L.
%
function [L,A] = mylu(A)

  n = length(A);
  L = eye(n);
  for j=1:n-1

    % Form vector of multipliers
    L(j+1:n,j) = A(j+1:n,j)/A(j,j);

    % Apply Gauss transformation
    A(j+1:n,j) = 0;
    A(j+1:n,j+1:n) = A(j+1:n,j+1:n)-L(j+1:n,j)*A(j,j+1:n);

  end
\end{lstlisting}

If we look closely, we see that at each step, the locations where we
write the multipliers in $L$ are exactly the same locations where we
introduce zeros in $A$!  Thus, we can re-use the storage space for $A$
to store $L$ (except for the diagonal ones, which are implicit) and
$U$.  Using this strategy, we have the following code:
\begin{lstlisting}
%
% Overwrite A with L and U factors
%
function [A] = mylu(A)
  n = length(A);
  for j=1:n-1
    A(j+1:n,j) = A(j+1:n,j)/A(j,j);
    A(j+1:n,j+1:n) = A(j+1:n,j+1:n) - A(j+1:n,j)*A(j,j+1:n);
  end
\end{lstlisting}
If we wanted to extract the $L$ and $U$ factors explicitly, we could
then do
\begin{lstlisting}
  LU = mylu(A);
  L = eye(length(A)) + tril(A,-1);
  U = triu(A);
\end{lstlisting}

The bulk of the work at step $j$ of the elimination algorithm is in
the computation of a rank-one update to the trailing submatrix.
How much work is there in total?  In eliminating column $j$, we do
$(n-j)^2$ multiplies and the same number of subtractions; so in all,
the number of multiplies (and adds) is
\[
  \sum_{j=1}^{n-1} (n-j)^2 = \sum_{k=1}^{n-1} k^2 = \frac{1}{6} n^3 + O(n^2)
\]
We also perform $O(n^2)$ divisions.  Thus, Gaussian elimination, like
matrix multiplication, is an $O(n^3)$ algorithm operating on $O(n^2)$ data.

\section{Schur complements}

The idea of expressing a step of Gaussian elimination as a low-rank
submatrix update turns out to be sufficiently useful that we give it
a name.  At any given step of Gaussian elimination, the trailing
submatrix is called a {\em Schur complement}.  We investigate the
structure of the Schur complements by looking at an $LU$
factorization in block 2-by-2 form:
\[
  \begin{bmatrix}
    A_{11} & A_{12} \\
    A_{21} & A_{22}
  \end{bmatrix} =
  \begin{bmatrix}
    L_{11} & 0 \\
    L_{21} & L_{22}
  \end{bmatrix}
  \begin{bmatrix}
    U_{11} & U_{12} \\
        0 & U_{22}
  \end{bmatrix} =
  \begin{bmatrix}
    L_{11} U_{11} & L_{11} U_{12} \\
    L_{21} U_{11} & L_{22} U_{22} + L_{21} U_{12}
  \end{bmatrix}.
\]
We can compute $L_{11}$ and $U_{11}$ as $LU$ factors of the leading
sub-block $A_{11}$, and
\begin{align*}
  U_{12} &= L_{11}^{-1} A_{12} \\
  L_{21} &= A_{21} U_{11}^{-1}.
\end{align*}
What about $L_{22}$ and $U_{22}$?  We have
\begin{align*}
  L_{22} U_{22}
  &= A_{22} - L_{21} U_{12} \\
  &= A_{22} - A_{21} U_{11}^{-1} L_{11}^{-1} A_{12} \\
  &= A_{22} - A_{21} A_{11}^{-1} A_{12}.
\end{align*}
This matrix $S = A_{22} - A_{21} A_{11}^{-1} A_{12}$ is the block analogue
of the rank-1 update computed in the first step of the standard
Gaussian elimination algorithm.

For our purposes, the idea of a Schur complement is important because
it will allow us to write blocked variants of Gaussian elimination ---
an idea we will take up in more detail now.


\section{Block Gaussian elimination}

Just as we could rewrite matrix multiplication in block form, we can also
rewrite Gaussian elimination in block form.  For example, if we want
\[
  \begin{bmatrix} A_{11} & A_{12} \\ A_{21} & A_{22} \end{bmatrix} =
  \begin{bmatrix} L_{11} & 0 \\ L_{21} & L_{22} \end{bmatrix}
  \begin{bmatrix} U_{11} & U_{12} \\ 0 & U_{22} \end{bmatrix}
\]
then we can write Gaussian elimination as:
\begin{enumerate}
\item
  Factor $A_{11} = L_{11} U_{11}$.
\item
  Compute $L_{21} = A_{21} U_{11}^{-1}$ and $U_{12} = L_{11}^{-1} A_{12}$.
\item
  Form the Schur complement $S = A_{22} - L_{21} U_{12}$ and factor
  $L_{22} U_{22} = S$.
\end{enumerate}

This same idea works for more than a block 2-by-2 matrix.
Suppose {\tt idx} is a \matlab\ vector that indicates the first index
in each block of variables, so that block $A_{IJ}$ is extracted as
\begin{lstlisting}
  I = idx(i):idx(i+1)-1;
  J = idx(j):idx(j+1)-1;
  A_IJ = A(idx(i):idx(i+1)-1, idx(j):idx(j+1)-1);
\end{lstlisting}
Then we can write the following code for block LU factorization:
\begin{lstlisting}
  M = length(idx)-1;          % Number of blocks
  for j = 1:M
    J    = idx(j):idx(j+1)-1; % Indices for block J
    rest = idx(j+1):n;        % Indices after block J
    A(J,J) = lu(A(J,J));      % Factor submatrix A_JJ

    % Extract L and U (this could be implicit)
    L_JJ   = tril(A(J,J),-1) + eye(length(J));
    U_JJ   = triu(A(J,J));

    % Compute block column of L and row of U, Schur complement
    A(rest,J) = A(rest,J)/U_JJ;
    A(J,rest) = L_JJ\A(J,rest);
    A(rest,rest) = A(rest,rest)-A(rest,J)*A(J,rest);
  end
\end{lstlisting}

As with matrix multiply, thinking about Gaussian elimination in this
blocky form lets us derive variants that have better cache efficiency.
Notice that all the operations in this blocked code involve matrix-matrix
multiplies and multiple back solves with the same matrix.  These routines
can be written in a cache-efficient way, since they do many floating point
operations relative to the total amount of data involved.

Though some of you might make use of cache blocking ideas in your own
work, most of you will never try to write a cache-efficient Gaussian
elimination routine of your own.  The routines in LAPACK and \matlab
(really the same routines) are plenty efficient, so you would most
likely turn to them.  Still, it is worth knowing how to think about
block Gaussian elimination, because sometimes the ideas can be specialized
to build fast solvers for linear systems when there are fast solvers for
sub-matrices

For example, consider the {\em bordered} matrix
\[
  A = \begin{bmatrix} B & W \\ V^T & C \end{bmatrix},
\]
where $B$ is an $n$-by-$n$ matrix for which we have a fast
solver and $C$ is a $p$-by-$p$ matrix, $p \ll n$.
We can factor $A$ into a product of {\em block} lower and
upper triangular factors with a simple form:
\[
  \begin{bmatrix} B & W \\ V^T & C \end{bmatrix} =
  \begin{bmatrix} B   & 0 \\ V^T & L_{22} \end{bmatrix}
  \begin{bmatrix} I & B^{-1} W \\ 0 & U_{22} \end{bmatrix}
\]
where $L_{22} U_{22} = C-V^T B^{-1} W$ is an ordinary (small) factorization
of the trailing Schur complement.  To solve the linear system
\[
  \begin{bmatrix} B & W \\ V^T & C \end{bmatrix}
  \begin{bmatrix} x_1 \\ x_2 \end{bmatrix} =
  \begin{bmatrix} b_1 \\ b_2 \end{bmatrix},
\]
we would then run block forward and backward substitution:
\begin{align*}
  y_1 &= B^{-1} b_1 \\
  y_2 &= L_{22}^{-1} \left( b_2 - V^T y_1 \right) \\
\\
  x_2 &= U_{22}^{-1} y_2 \\
  x_1 &= y_1-B^{-1} (W x_2)
\end{align*}


\section{Perturbation theory}

Previously, we described a general error analysis strategy: derive
forward error bounds by combining a sensitivity estimate (in terms of
a {\em condition number}) with a {\em backward} error analysis that
explains the computed result as the exact answer to a slightly
erroneous problem.  To follow that strategy here, we need the
sensitivity analysis of solving linear systems.

Suppose that $Ax = b$ and that $\hat{A} \hat{x} = \hat{b}$,
where $\hat{A} = A + \delta A$, $\hat{b} = b + \delta b$,
and $\hat{x} = x + \delta x$.  Then
\[
  \delta A \, x + A \, \delta x + \delta A \, \delta x = \delta b.
\]
Assuming the delta terms are small, we have the linear approximation
\[
  \delta A \, x + A \, \delta x \approx \delta b.
\]
We can use this to get $\delta x$ alone:
\[
  \delta x \approx A^{-1} (\delta b - \delta A \, x);
\]
and taking norms gives us
\[
  \|\delta x\| \lesssim \|A^{-1}\| (\|\delta b\| + \|\delta A\| \|x\|).
\]
Now, divide through by $\|x\|$ to get the relative error in $x$:
\[
  \frac{\|\delta x\|}{\|x\|} \lesssim \|A\| \|A^{-1}\|
    \left( \frac{\|\delta A\|}{\|A\|} + \frac{\|\delta b\|}{\|A\|\|x\|} \right).
\]
Recall that $\|b\| \leq \|A\|\|x\|$ to arrive at
\[
  \frac{\|\delta x\|}{\|x\|} \lesssim \kappa(A)
    \left( \frac{\|\delta A\|}{\|A\|} + \frac{\|\delta b\|}{\|b\|} \right),
\]
where $\kappa(A) = \|A\| \|A^{-1}\|$.  That is, the relative error in
$x$ is (to first order) bounded by the condition number times the relative
errors in $A$ and $b$.  We can go beyond first order using Neumann
series bounds -- but perhaps not today.

\section{Residual good!}

The analysis in the previous section is pessimistic in that it gives us
the worst-case error in $\delta x$ for {\em any} $\delta A$ and $\delta b$.
But what if we are given data that behaves better than the worst case?

If we know $A$ and $b$, a reasonable way to evaluate an approximate
solution $\hat{x}$ is through the residual $r = b-A\hat{x}$.  The
approximate solution satisfies
\[
  A \hat{x} = b + r,
\]
so if we subtract of $Ax = b$, we have
\[
  \hat{x}-x = A^{-1} r.
\]
We can use this to get the error estimate
\[
  \|\hat{x}-x\| = \|A^{-1}\| \|r\|;
\]
but for a given $\hat{x}$, we also actually have a prayer of {\em evaluating}
$\delta x = A^{-1} r$ with at least some accuracy.
%
It's worth pausing to think how novel this situation is.
Generally, we can only {\em bound} error terms.
If I tell you ``my answer is off by just about 2.5,''
you'll look at me much more sceptically than if I tell
you ``my answer is off by no more than 2.5,'' and reasonably so.
After all, if I knew that my answer was off by nearly 2.5, why wouldn't
I add 2.5 to my original answer in order to get something closer to truth?
This is exactly the idea behind {\em iterative refinement}:
\begin{enumerate}
\item
  Get an approximate solution $A \hat{x}_1 \approx b$.
\item
  Compute the residual $r = b-A\hat{x}_1$ (to good accuracy).
\item
  Approximately solve $A \, \delta x_1 \approx r$.
\item
  Get a new approximate solution $\hat{x}_2 = \hat{x}_1+\delta x_1$;
  repeat as needed.
\end{enumerate}


\end{document}
