\documentclass[12pt, leqno]{article}
\usepackage{amsthm}
\input{common}

\newtheorem{lemma}{Lemma}

\begin{document} \hdr{Proj 1: Harmonious Learning}

There are many problems that involve optimizing some objective
function by making local adjustments to a structure or graph.
For example:
\begin{itemize}
\item If we want to reinforce a truss with a limited budget, where
  should we add new beams (or strengthen old ones)?
\item After a failure in the power grid, how should lines be either
  taken out of service or put in service to ensure no other lines
  are overloaded?
\item In a road network, how will road closures or rate-limiting of
  on-ramps affect congestion (for better or worse)?
\item In a social network, which edges are most critical to
  spreading information or influence to a target audience?
\end{itemize}

For our project, we will consider a simple method for {\em graph
interpolation}.  We are given a (possibly weighted) undirected graph on
$n$ nodes, and we wish to determine some real-valued numerical property
at each node.  Given values at a few of the nodes, how should we fill in
the remaining values?  A natural approach that is used in some
semi-supervised machine learning approaches is to fill in the remaining
values by assuming that the value at an unlabeled node $i$ is the
(possibly weighted) average of the values at all neighbors of the node.
In this project, we will see how to quickly solve this problem, and how
to efficiently evaluate the sensitivity with respect to different types
of changes in the setup.  Of course, in the process we also want to
exercise your knowledge of linear systems, norms, and the like!

\section*{Logistics}

{\bf You are encouraged to work in pairs on this project.}  You should
produce short report addressing the analysis tasks, and a few
short codes that address the computational tasks.  You may
use any MATLAB or Octave functions you might want.

Most of the code in this project will be short, but that does not make
it easy.  You should be able to convince both me and your partner that
your code is right.  A good way to do this is to test thoroughly.
Check residuals, compare cheaper or more expensive ways of computing
the same thing, and generally use the computer to make sure you don't
commit silly errors in algebra or coding.  You will also want to make
sure that you satisfy the efficiency constraints stated in the tasks.

\section*{Background}

The (combinatorial) {\em graph Laplacian} matrix occurs often when
using linear algebra to analyze graphs.  For an undirected graph on
vertices $\{1, \ldots, n\}$, the weighted graph
Laplacian $L \in \bbR^{n \times n}$ has entries
\[
  l_{ij} = \begin{cases}
    -w_{ij}, & \mbox{ if } (i,j) \mbox{ an edge with weight } w_i \\
    d_{i} = \sum_k w_{ik}, & \mbox{i = j} \\
    0, & \mbox{otherwise}.
  \end{cases}
\]
The unweighted case corresponds to $w_{ij} = -1$ and $d$ equal to the
node degree.

In our problem, we seek to solve problems of the form
\[
  \begin{bmatrix}
    L_{11} & L_{12} \\
    L_{21} & L_{22}
  \end{bmatrix}
  \begin{bmatrix} u_1 \\ u_2 \end{bmatrix} =
  \begin{bmatrix} u \\ r_2 \end{bmatrix}
\]
where the leading indices correspond to nodes in the graph at which
$u$ must be inferred (i.e.~$u_1$ is an unknown) and the remaining
indices correspond to nodes in the graph at which $u$ is specified
(i.e.~$u_2$ is known, though $r_2$ is not).  Note that if $i$ is an
index in the first block, then the equation at row $i$ specifies that
\[
  u_i = \frac{1}{d_i} \sum_{(i,j) \in \mathcal{E}} w_{ij} u_j,
\]
i.e.~the value at $i$ is a weighted average of the neighboring values.

\section*{Your tasks}

We will use the California road network data from the SNAP data set; to
retrieve it, download the {\tt roadNet-CA.txt} file from the class web
page and use the loader script included to read in the topology and form
the graph Laplacian. This is a big enough network that you will {\em
not} want to form the graph Laplacian or related matrices in dense form.
On the other hand, because it is a moderate-sized planar graph, sparse
Cholesky factorization on $L$ will work fine.

The {\tt README} file included with the code describes the baseline
code, with places where you should fill in additional code marked by
{\tt TASK} comments.  We have provided a testing script as part of the
baseline code.

\paragraph*{Task 1}
Fill in {\tt ginterp\_eval0} with code to solve the
graph interpolation problem.  The code already computes index vectors
{\tt Ia} and {\tt Ib} corresponding to the free variables (which
must be computed) and the boundary values (which are known).  On
my laptop, this takes about four seconds.  You should be careful with
your parentheses, and you should {\em not} attempt an explicit inverse
(which will probably crash your machine).

\paragraph*{Task 2}
As a set-up for the next steps, we are going to separate the solve
in the first task into two components.  The {\tt ginterp\_factor}
routine will compute a sparse Cholesky factorization, while the
{\tt ginterp\_eval} routine will use that factorization to solve
the linear system.  Make sure that you use the version of {\tt chol}
that returns a permutation for sparsity!

\paragraph*{Task 3}
Suppose now that we wish to incrementally add new specified values
to the system.  One way to do this would be to update which nodes
are free and which have specified values, then recompute the
factorization.  We will try a different approach, which is to solve
a {\em bordered system}
\[
  \begin{bmatrix}
    L_{11} & L_{12} & B_1 \\
    L_{21} & L_{22} & B_2 \\
    B_1^T & B_2^T & C
  \end{bmatrix}
  \begin{bmatrix}
    u_1 \\ u_2 \\ w
  \end{bmatrix} =
  \begin{bmatrix}
    0 \\ r_2 \\ f
  \end{bmatrix}.
\]
To enforce additional boundary conditions, we use each column of $B_1$
to indicate a node to constrain, and let the corresponding entry of $f$
be the value at that node.  The $B_2$ and $C$ matrices will be zero
in this case.  You can complete this functionality by filling in the
helper function {\tt ginterp\_bsys} and extending {\tt ginterp\_eval}.
Your solution should not do any new sparse matrix factorizations, but
will involve $O(k)$ new solves with the pre-computed factorization of
$L_{11}$, where $k$ is the number of new boundary conditions.

\paragraph*{Task 4}
Beyond using extra variables to enforce new boundary conditions, we
can also use them to update the edge weights in the graph.  For
example, consider an increase of $s$ to the weight of edge $(i,j)$
in the graph.  The Laplacian for the new graph would be
\[
  L' = L + s (e_i-e_j) (e_i-e_j)^T,
\]
and we can write $L'u$ as $Lu + (e_i-e_j) \gamma$ where
$\gamma = s (e_i-e-j)^T u$.  Using this observation, extend the
{\tt ginterp\_bsys} command to form a bordered system that incorporates
edge weight modifications as well as additional boundary conditions,
all without re-computing any large sparse factorizations.

\paragraph*{Task 5}
Using bordered systems lets us recompute the solution quickly after we
adjust the edge weights.  But what if we want to compute the sensitivity
of the value at some target node to small changes to {\em any} of the
edges?  That is, for a target node $k$, we think of $u_k$ as a function
of all the edge weights, and compute the sparse sensitivity matrix
\[
  S_{ij} =
  \begin{cases}
    \frac{\partial u_k}{\partial w_{ij}}, & (i,j) \in \mathcal{E} \\
    0, & \mbox{otherwise}.
  \end{cases}
\]
Assuming the $u$ vector has already been computed, the sensitivity
computation requires constant work per edge after one additional
linear solve.  Fill in {\tt ginterp\_deriv} to carry out this
computation.  Note that your code should ideally use the bordered
system formalism to incorporate any new boundary conditions or
edge updates added to the system since the last factorization.

\section*{Afternotes}

\begin{enumerate}
\item Almost all the tasks in this project boil down to block factorization,
  using sparse factorizations, and sensitivity analysis.  Nothing should take
  many lines of code if you do it right.  Nonetheless, the project is
  not trivial --- so ask questions!  Office hours and Piazza are your friends.
\item If you follow the intended path, none of the computations should
  be too expensive.  However, the road network is large enough that
  you will cause yourself serious pain if you attempt to form it as a
  dense matrix.  You may also run into trouble if you attempt a direct
  factorization without permuting for sparsity.
\end{enumerate}

\end{document}
